
\documentclass[12pt]{letter}

\usepackage{color}

\setlength{\textwidth}{17cm}
\setlength{\textheight}{25cm}
\setlength{\topmargin}{-1cm}
\setlength{\leftmargin}{-1cm}

\begin{document}

proviamo a scrivere una lettera\\
{\rm di solito usiamo il Roman}\\
{\it oppure l'italico }\\
{\bf oppure utilizziamo il grassetto}\\
{\bf\it o una combinazione dei due }\\
{\tiny e cambiamo la dimensione dei fonts, da piccoli}\\
{\scriptsize lentamente pi\`u grandi}\\
{\footnotesize ancora pi\'u grandi}\\
{\normalsize di nuovo normali}\\
{\large e poi sempre} {\Large pi\`u grandi} {\LARGE fino a diventare}\\
{\huge letteralmente} {\Huge ENORMI}\\

\clearpage
%%%%%%%%%%%%%%%%%%%%%%%%%%%%%%%%%%%%%%%%%%%%%%%%%%%%%%%%%%%%%%%%%%%%%%%%%%%%%%%

i comandi in LaTeX sono preceduti da back-slash ovvero dal simbolo 
$\backslash$\\
L'accentazione delle lettere avviene facendo precedere
l'accento alla lettera su cui deve collocarsi.\\
p.es: \`e oppure \'e oppure \^e oppure \`u oppure \"u 
oppure \~n\\
nel caso della lettera i, che ha per default il puntino, bisogna levare 
il puntino, ottenendo \`\i \\
alcuni caratteri sono "speciali" poich\'e servono come istruzioni in LaTeX
per farli comparire \`e necessario utilizzare $\backslash$\\
tra i pi\`u comuni ci sono il dollaro \$ , la \& e le parentesi 
graffe \{ e \}

\clearpage
%%%%%%%%%%%%%%%%%%%%%%%%%%%%%%%%%%%%%%%%%%%%%%%%%%%%%%%%%%%%%%%%%%%%%%%%%%%%%%%

\`e possibile inserire spazi a piacimento\\[2cm]
sia verticalmente che \hskip5cm orizzontalmente\\
e inoltre anche con segno negativo\\[1cm]
\`e possibile 
\parbox[t]{15cm}
{individuare dei riquadri entro cui inserire un\\
paragrafo\\
per dargli maggiore evidenza
}\\
oppure
\hskip8cm
\parbox[t]{5cm}
{per impaginare un indirizzo\\
Via Celoria 16\\
20133 Milano}

sempre nella classe dei box ci sono
\framebox[5cm][t]{ciao ciao}
quelli con cui si possono evidenziare delle parole

\begin{center}
per i titoli\\
{\large\bf a volte fa piacere}\\
{\Huge\it avere la possibilit\`a di centrare}\\
{\tiny la frase} 
\end{center}


\clearpage
%%%%%%%%%%%%%%%%%%%%%%%%%%%%%%%%%%%%%%%%%%%%%%%%%%%%%%%%%%%%%%%%%%%%%%%%%%%%%%%

a volte \`e importante scrivere una tabella

\begin{tabular}{lcrl}
 & Qui & Quo & Qua \\
corsetta & 9 & 11 & 123\\
salto & 321 & 99  & 1\\
\end{tabular}


e ovviamente pu\`o essere utile tracciare anche delle linee

\begin{tabular}{|l|c|r|l|}
\hline
 & Qui & Quo & Qua \\
\hline
corsetta & 9 & 11 & 123\\
\hline
salto & 321 & 99  & 1\\
\hline
\end{tabular}



\end{document}
